\input{@resources/tex-add-search-paths}  % allow latex to find custom stuff
% -*- mode: LaTeX; TeX-PDF-mode: t; -*-
% this file is in the root directory so . is the path to the root 
\providecommand{\econtexRoot}{}\renewcommand{\econtexRoot}{.}

\documentclass[BufferStockTheory]{subfiles}
   

\newcommand{\subname}{Introduction}   % Needed for generic bib to work
%\providecommand{\ApndxDir}{}\renewcommand{\ApndxDir}{\econtexRoot/Appendices}
\providecommand{\EqDir}{}\renewcommand{\EqDir}{\econtexRoot/Equations}
\providecommand{\TableDir}{}\renewcommand{\TableDir}{\econtexRoot/Tables}
\providecommand{\FigDir}{}\renewcommand{\FigDir}{\econtexRoot/Figures}
\providecommand{\LaTeXInputs}{}\renewcommand{\LaTeXInputs}{\econtexRoot/@resources/texlive/texmf-local/tex/latex}
\providecommand{\LaTeXGenerated}{}\renewcommand{\LaTeXGenerated}{\econtexRoot} % not worth trying to put generated files in a subdir
\providecommand{\ResourcesDir}{}\renewcommand{\ResourcesDir}{\econtexRoot/@resources}
\providecommand{\LtxDir}{}\renewcommand{\LtxDir}{}
 % get directory macros
\usepackage{econark-ifsubfile}
\usepackage{econark-xrsetup}

% if it's being compiled standalone, set up identity of the external
\compilingasstandalone{
  \xrsetup{\econtexRoot/\texname}
}

% \begin{document}

% Section~\ref{subsubsec:deatonIsLimit})

% The first panel of Table~\ref{table:Required} summarizes: The PF-Unconstrained model has a nondegenerate limiting solution if we impose the {\RIC} and {\FHWC} (these conditions are necessary as well as sufficient).
% Together the {\PFFVAC} and the {\FHWC} imply the {\RIC}.

% \cite{carroll:death}

% \compilingasstandalone{
%   \bibliography{Introduction}
%   }
% \end{document}\endinput

\newcommand{\thankstext}{
  The paper's results \href{https://\owner.github.io/nbreproduce}{can be automatically reproduced} using the {\ARKurl} toolkit, which can be cited per our references (\cite{carroll_et_al-proc-scipy-2018}); for reference to the toolkit itself see \href{https://econ-ark.org/acknowledging}{Acknowledging Econ-ARK}.
Thanks to the \href{https://consumerfinance.gov}{Consumer Financial Protection Bureau} for funding the original creation of the {\ARKurl} toolkit; and to the \href{https://sloan.org}{Sloan Foundation} for funding Econ-ARK's \href{https://sloan.org/grant-detail/8071}{extensive åfurther development} that brought it to the point where it could be used for this project.
The toolkit can be cited with its digital object identifier, \href{https://doi.org/10.5281/zenodo.1001067}{10.5281/zenodo.1001067}, as is done in the paper's own references as \cite{carroll_et_al-proc-scipy-2018}.
Thanks to Will Du, James Feigenbaum, Joseph Kaboski, Miles Kimball, Qingyin Ma, Misuzu Otsuka, Damiano Sandri, John Stachurski, Adam Szeidl, Alexis Akira Toda, Metin Uyanik, Mateo Vel\'asquez-Giraldo, Weifeng Wu,  Jiaxiong Yao, and Xudong Zheng for comments on earlier versions of this paper, John Boyd for help in applying his weighted contraction mapping theorem, Ryoji  Hiraguchi for extraordinary mathematical insight that improved the  paper greatly, David Zervos for early guidance to the literature, and participants in a seminar at the Johns Hopkins University, a presentation at the 2009 meetings of the Society of Economic Dynamics for their insights, and at a presentation at the Australian National University.}

% \newcommand{\inframe}{https://\owner.github.io/BufferStockTheory/BufferStockTheory3.html}

% Embed metadata
\hypersetup{pdfauthor={Christopher D Carroll <ccarroll@jhu.edu>},
  pdftitle={Theoretical Foundations of Buffer Stock Saving},
  pdfkeywords={Precautionary saving, buffer-stock saving, consumption, marginal propensity to consume, permanent income hypothesis, income fluctuation problem},
  pdfnewwindow=true,
  pdfcreator = {ccarroll@jhu.edu}
}

\begin{document}

% Attempted to make all lines used for Web version contain {Web} (or version with only single curly brace at end) so can be removed with sed
\ifthenelse{\boolean{Web}}{    % {Web}
  \renewcommand{\versn}{Web}     % Too hard to figure out passing -output-directory through make4ht through htlatex, so web version is compiled with junk files in main directory
  \renewcommand{\rootFromOut}{.} % {Web}
}{}  % {Web}


% Tiny info header at top tracks git commit

\title{Theoretical Foundations of \\ Buffer Stock Saving}

\author{Christopher D Carroll\authNum ~and Akshay Shanker\authNum}

\keywords{Precautionary saving, buffer stock saving, marginal propensity to consume, permanent income hypothesis, income fluctuation problem}

\jelclass{D81, D91, E21 \par
  \href{https://econ-ark.org}{\includegraphics{@resources/econ-ark/PoweredByEconARK}}
}

\date{\today}
% \renewcommand{\forcedate}{April 24, 2023}\date{\forcedate}

\maketitle
\hypertarget{abstract}{}
\begin{abstract}
  This paper builds the foundations for a rigorous and intuitive understanding of `buffer stock' saving, behaviour that emerges in \cite{bewleyPIH}-like economies with a wealth target, pairing each theoretical result with quantitative illustrations.  After describing conditions under which a consumption function exists, the paper articulates stricter `Growth Impatience' conditions that guarantee alternative forms of `target' saving --- either at the population level, or for individual consumers.  Together, the numerical tools and analytical results constitute a comprehensive toolkit for understanding buffer stock models.
\end{abstract}

% Various resources
\hypertarget{links}{}

\begin{footnotesize}
  \parbox{0.9\textwidth}{
    \begin{center}
      \begin{tabbing}
        \texttt{~~~\REMARK:~} \= \= \texttt{\url{https://econ-ark.org/materials/bufferstocktheory}} \\ 
        \texttt{Dashboard:~} \> \> \texttt{\href{https://econ-ark.org/materials/bufferstocktheory}{Click `Launch Dashboard' Button}} \\
        \texttt{~~~~~html:~} \> \> \texttt{\href{https://\owner.github.io/BufferStockTheory/}{https://\owner.github.io/BufferStockTheory/}} \\ % Owner is defined in Resources/owner.tex
        \texttt{~~~~~~PDF:~} \> \> \texttt{\href{https://github.com/\owner/BufferStockTheory/blob/master/BufferStockTheory.pdf}{BufferStockTheory.pdf}} \\ 
        \texttt{~~~Slides:~} \> \> \texttt{\href{https://github.com/\owner/BufferStockTheory/blob/master/LaTeX/BufferStockTheory-Slides.pdf}{BufferStockTheory-Slides.pdf}} \\
        % \texttt{~Appendix:~} \> \> \texttt{\url{https://\owner.github.io/BufferStockTheory/BufferStockTheory3.html#Appendices}}    \\
        \texttt{~~~GitHub:~} \> \> \texttt{\href{https://github.com/\owner/BufferStockTheory}{https://github.com/\owner/BufferStockTheory}} \\
      \end{tabbing}
    \end{center}
    % The \href{https://econ-ark.org/materials/bufferstocktheory\?dashboard}{dashboard} lets users see consequences of alternative parameters in an interactive framework.
  } % end \parbox{\textwidth}
\end{footnotesize}

\begin{authorsinfo}
  \name{Contact: \href{mailto:ccarroll@jhu.edu}{\texttt{ccarroll@jhu.edu}}, Department of Economics, 590 Wyman Hall, Johns Hopkins University, Baltimore, MD 21218, \url{https://www.econ2.jhu.edu/people/ccarroll}, and National Bureau of Economic Research.}\name{Contact: \href{mailto:akshay.shanker@unsw.edu.au}{\texttt{akshay.shanker@unsw.edu.au}}, Department of Economics, University of New South Wales, Sydney, Australia and Australian Research Council (ARC) Center for Excellence in Population and Ageing Research (CEPAR).}
\end{authorsinfo}

\pagenumbering{gobble} % Prevent numbering for pages including the TOC and title page

\ifthenelse{\boolean{Web}}{
}{
  \begin{minipage}{0.9\textwidth}
    \footnotesize The paper's results \href{https://\owner.github.io/nbreproduce}{can be automatically reproduced} using the {\ARKurl} toolkit by executing the \href{https://econ-ark.org/materials/bufferstocktheory}{notebook}; for reference to the toolkit itself see \href{https://econ-ark.org/acknowledging}{Acknowleding Econ-ARK}.  Thanks to the \href{https://consumerfinance.gov}{Consumer Financial Protection Bureau} for funding the original creation of the {\ARKurl} toolkit; and to the \href{https://sloan.org}{Sloan Foundation} for funding Econ-ARK's \href{https://sloan.org/grant-detail/8071}{extensive further development} that brought it to the point where it could be used for this project.  The toolkit can be cited with its digital object identifier, \href{https://doi.org/10.5281/zenodo.1001067}{https://doi.org/10.5281/zenodo.1001067}, as is done in the paper's own references as \cite{carroll_et_al-proc-scipy-2018}.  Thanks to Will Du, James Feigenbaum, Joseph Kaboski, Miles Kimball, Qingyin Ma, Misuzu Otsuka, Damiano Sandri, John Stachurski, Adam Szeidl, Alexis Akira Toda, Metin Uyanik, Mateo Vel\'asquez-Giraldo, Weifeng Wu,  Jiaxiong Yao, and Xudong Zheng for comments on earlier versions of this paper, John Boyd for help in applying his weighted contraction mapping theorem, Ryoji  Hiraguchi for extraordinary mathematical insight that improved the  paper greatly, David Zervos for early guidance to the literature, and participants in a seminar at the Johns Hopkins University, a presentation at the 2009 meetings of the Society of Economic Dynamics for their insights, and at a presentation at the Australian National University.
  \end{minipage}

  \titlepagefinish\pagebreak
  \let\LaTeXStandardContentsName\contentsname
  \renewcommand{\contentsname}{}
  \tableofcontents
  \pagebreak

  \medskip\medskip
  \begin{minipage}{0.9\textwidth}
    \listoffigures 
  \end{minipage}

  \medskip\medskip
  \begin{minipage}{0.9\textwidth}
    \listoftables
  \end{minipage}
} % {Web}


\pagebreak
\hypertarget{Introduction}{}
\section{Introduction}\label{sec:intro}
\setcounter{page}{0}\pagenumbering{arabic}

% Part 1: Opening
% - Define buffer stock saving
% - Outline its significance
% - Describe the motivation and contribution of the paper
% 
% Todo:
% -


The precautionary motive to save springs from the fact that extra resources give a consumer more ability to buffer spending against income shocks.\footnote{\cite{CarrollKimballPSPW}.}  A consumer who, in the absence of shocks, would be `impatient' enough to want to spend down resources, will (when there are shocks to worry about) experience an intensifying precautionary motive as their buffering capacity shrinks.
If resources fall far enough, the consequence may be to make `prudence' (\cite{kimball:standardra}) strong enough to counterbalance impatience.
A consumer whose behavior is governed by this competition between impatience and prudence has been described, starting with \cite{deatonLiqConstr}, as engaging in `buffer stock saving.'


The logic of buffer stock saving underpins key findings in heterogeneous-agent macroeconomics.
For example, it can explain why, during the Great Recession, middle-class consumers cut spending more than the poor or the rich~\citep{kmpHandbook}.
Buffer stock saving also can explain why consumption growth tracks income growth over much of the life cycle \citep{carrollBSLCPIH}, rather than being determined solely by preferences and interest rates as Irving~\cite{fisherInterestTheory} proposed (and as log-linearized Euler equations erroneously suggest (\cite{carroll:death}).


Despite the central role that buffer stock saving plays in HA-macro, the literature lacks a formal theory mapping the conditions under which such behavior emerges.
This paper establishes the required foundations by explaining the circumstances under which preferences, income growth, uncertainty, and the interest rate imply the existence of buffer-stock saving `targets' both at the individual level and in the aggregate.

% Part 2: Gap in literature 
% - State gap in technical literature 
% - State technical problem 
% - List main contributions 
% - Describe the motivation and contribution of the paper
% 
% Todo:

We formalize buffer stock saving using the Friedman-Muth(-Zeldes) income fluctuation model, incorporating realistic transitory and permanent shocks~\citep{friedmanATheory, muthOptimal, zeldesStochastic},\footnote{By which we mean, calibrated to micro empirical evidence.}  constant relative risk aversion (CRRA) utility, and a `natural borrowing constraint' of the kind first employed by \cite{zeldesStochastic}.\footnote{A natural borrowing constraint is the maximum amount a consumer will be willing to borrow under any circumstances.
See~\cite{carrollBSLCPIH} or~\cite{gpLifeCycle} for arguments that models with only `natural' constraints (see below) match a wide variety of facts; for a model with explicit constraints that produces very similar results, see, e.g.~\cite{Cagetti}.
They are analytically convenient as the consumption function becomes twice continuously differentiable and also becomes arbitrarily close (cf.\ Section~\ref{subsubsec:deatonIsLimit}) to less tractable models with artificial liquidity constraints.}\textsuperscript{,}\footnote{The model permits separate transitory and permanent shocks (\textit{a la}~\cite{muthOptimal}) and permanent growth in income, which a large empirical literature finds to be of dominant importance in microdata.
For example, MaCurdy~\citeyearpar{macurdyTimeseries}; Abowd and Card~\citeyearpar{acCovariance}; Carroll and Samwick~\citeyearpar{csNature}; Jappelli and Pistaferri~\citeyear{jpCins}; et.
seq.
Much of the literature instead incorporates highly `persistent' but not completely permanent shocks, but~\cite{dhmImproving} show that when measurement problems are handled correctly, admin data yield serial correlation coefficients $0.98-1.00$; and~\cite{dmHowMuch} suggests that survey data support the same conclusion.}  In an infinite-horizon (or a \cite{blanchardFinite} perpetual youth) version of this framework, we establish and explain the economic implications of two main results.


Our first contribution identifies conditions under which non-degenerate\footnote{Real valued solutions with strictly positive consumption as the planning horizon becomes arbitrarily distant.} infinite-horizon solutions exist without any `artificial' liquidity constraint.
The literature since \cite{bewleyPIH} and proceeding all the way through recent contributions by \cite{mstIncFluct,maUnboundedDP} has imposed an artificial liquidity constraint strictly tighter than the `natural' constraint because this guarantees that the dynamic programming problem has a compact stationary Bellman operator (in the sense articulated below), permitting the use of contracting mapping arguments to show existence (\cite{stachurski2022}).
Our proof of the case without artificial constraints holds even if a compact stationary Bellman operator may not exist.

Once we have established the existence of a non-degenerate solution, the second (and most important) contribution of the paper is to identify conditions under which buffer stock `targets' exist, for individual consumers or in the aggregate.
Later we show that the `artificial' case is a limiting version of the natural case.



% Part 4: State set of assumptions required for exisistence
% - State WRIC and also define APF
% - State FVAC and its role  
% - Relate Bellman existence to RIC 
% 
% Todo: Akshay: I feel there needs to be a bit more motivation of buffer stock stability above, it is heavily focused on non-degenrate solutions.


The existence of nondegenerate solutions and buffer stock targets turn out to depend on various `patience' conditions (combinations of parameters like the time preference rate, relative risk aversion, the rate of return, the underlying growth rate of income and the nature of income growth's stochastic elements).
The simplest of these conditions (and one familiar from a large literature) corresponds to what we call `absolute patience':  A consumer who exhibits `absolute impatience' is someone who, in the absence of a precautionary motive, would want to move future resources to the present to achieve a higher level of current consumption, because their pure rate of time preference makes them discount the future more than the interest rate encourages them to save (when $(\Rfree \DiscFac)<1$).
But, as \cite{szeidlInvariant} and \cite{maUnboundedDP} point out, the $(\Rfree \DiscFac)<1$ condition is neither necessary nor sufficient for the model to have a nondegenerate solution.


Our patience conditions are most easily understood by comparison with the perfect foresight case with no borrowing constraint.
For any given planning horizon, there will be a strictly positive marginal propensity to consume that optimally allocates current wealth between the present and the future.
Of course, extending the horizon increases the number of future periods over which a given level of wealth can be spread, increasing the appeal of saving.
The marginal propensity to save (and consume) in the limit depends on the term $\APFac=(\Rfree \DiscFac)^{1/\CRRA}$, which we designate as the consumer's \hyperlink{APFAC}{rate of absolute patience}.
The \hyperlink{APFAC}{rate of absolute patience} is a central concept in our paper and can be understood as the rate of consumption growth for a perfect foresight consumer with relative risk aversion of $\CRRA$.
In the perfect foresight case without borrowing constraints, what we show is necessary for a non-degenerate solution is `return impatience': $\APFac < \Rfree$.
This condition guarantees that as the horizon extends arbitrarily, the limiting solution is one in which the consumer is not so patient that (in the limit) they will allocate all their wealth to the future (consuming nothing now).\footnote{This condition is \textit{not} required for nondegeneracy when there is an artificial borrowing constraint, so long as the consumer would wish to bring future resources to the present by borrowing -- which occurs if income growth exceeds the interest rate.}

The first main result of the paper is that for our model with stochastic income shocks, the condition required for nondegeneracy is (surprisingly) \textit{weaker} than the condition required for a nondegenerate solution in the perfect foresight case: What is needed is \hyperlink{WRIC}{`weak return impatience.'} If a consumer is weak return impatient, then even if their wealth barely exceeds the limits imposed by the natural borrowing constraint, they will consume enough to make their expected resources fall.
In addition to weak return impatience, we show that in the presence of permanent growth, the standard `$\DiscFac<1$' requirement must be modified to take account of income growth uncertainty; we describe the new requirement as requiring \hyperlink{FVAC}{`finite value of autarky'} (where here we think of autarky as perpetual consumption of your permanent income).
If weak return impatience and finite value of autarky are satisfied, then we show that a nondegenerate solution  exists.

%Our method of proof uses a novel argument by utilizing the upper and lower bounds of consumers' marginal propensities of consume (MPCs) to show \textit{`per-period' Bellman operators} are well-defined contraction maps. The finite horizon value functions are then shown to converge to the nondegenerate solution.%\footnote{They are a Cauchy sequence in a complete weighted-norm space, converging to a nondegenerate solution as the planning horizon recedes.}

The introduction of uncertainty brings a precautionary motive that enhances the consumer's preference for saving.
So it is a surprise that in the stochastic setting, the `patience' condition required for nondegeneracy is weaker than that required for the perfect foresight unconstrained case.
The reason has to do with the other aspect of the introduction of uncertainty, which is the natural borrowing constraint imposed by the requirement that debts be repaid.
Just as in the perfect foresight case, if the consumer would want to borrow against future income that is growing faster than the rate at which it is discounted, the natural borrowing constraint does what the artificial borrowing constraint does in the perfect foresight case: It prevents too much borrowing.

% Part 3: State key methodological innovation for first contribution 
% - Use of MPC bounds 
% - How the proofs works breifly  
% - Bellman may not exist, yet VFs converge 
% 
% Todo:

% Part 3: State key methodological innovation for first contribution 
% - Use of MPC bounds 
% - How the proofs works breifly  
% - Bellman may not exist, yet VFs converge 
% 
% Todo:
%  - add reference to stachurski, maybe call them 'time-indexed' bellman operators. AS to AS: still tod this?

% Part 5: State set of assumptions required for individual stability 
% - Symmetry with exposition: it is all about APF and G or R
% - State GIC-Mod 
% - Intuition of GIC-Mod in terms of growth rates 
% 
% Todo:

Turning to our results on the buffer stock target, the requirement for the existence of an individual target is \hyperlink{GICMod}{`strong growth impatience,'} which prevents `normalized market resources' (the ratio of market resources to permanent income), $\mNrm$, from growing without bound.
In particular, strong growth impatience requires that as a consumer's resources grow, eventually a point will come at which the expected ratio of a consumer's \hyperlink{APFAC}{absolute patience} to the uncertainty-adjusted growth of permanent income is less than one.
Under the condition, as a consumer's normalized market resources approach infinity, the \textit{expected ratio} of market resources--incorporating optimal saving--to the uncertain permanent income growth factor, must eventually becomes less than one.
That is, normalized market resources eventually revert back toward a target.

% Part 6: State set of assumptions required for aggregate stability 
% - State GIC 
% - State aggregate implications 
% 
% Todo:


A weaker requirement, \hyperlink{GIC}{`growth impatience,'} ensures the existence of an aggregate buffer stock target even when individual target ratios are unbounded.
Growth impatience requires the \textit{ratio} of \hyperlink{APFacDefn}{absolute patience} to the \textit{expected} growth factor of permanent income to be less than one.
But recall that at the individual level, the ratio of wealth to permanent income might grow either as a result of an increase in wealth or as the result of a decrease in permanent income.
Idiosyncratic uncertainty in permanent income means that in any given year there will be a portion of consumers whose permanent income has declined.
Such consumers wll experience an increase  in the ratio of wealth to permanent income, even if their absolute wealth has stagnated or even fallen (but by less than the decline in permanent income).

Nonetheless, when \hyperlink{GIC}{`growth impatience'} holds, the ratio of average market resources to average permanent income converges back to a target.
As \cite{harmenbergInvariant} points out, a stationary distribution of market resources, weighted by permanent income still exists under growth impatience.
We develop the insight by \cite{harmenbergInvariant} and demonstrate that the contribution to aggregate consumption of consumers who accumulate unbounded resources diminishes as they receive a smaller and smaller measure of permanent income.
Thus in the aggregate, even with a fixed aggregate interest rate that differs from the time preference rate, a small open economy populated by buffer stock consumers has a balanced growth path in which growth rates of consumption, income, and wealth match the exogenous growth rate of permanent income (equivalent, here, to productivity growth).
In the terms of~\cite{schmitt2003closing}, buffer stock saving is an appealing method of `closing' a small open economy, because it requires no \textit{ad-hoc} assumptions.
Not even liquidity constraints.

\begin{comment}
  \footnote{The paper's insights are instantiated in the {\ARKurl} toolkit, whose \href{https://hark.readthedocs.io/en/stable/reference/ConsumptionSaving/ConsIndShockModel.html}{buffer stock saving module} flags parametric choices under which a problem is degenerate or under which stable ratios of wealth to income may not exist.}
\end{comment}

% Part 7: Now brings it all toghether on economic discussion of APF, G and R
% - GIC and RIC Tied together by FVAC
% - GIC required when RIC fails i.e. income growth balances patience 
% - low income growth requires impatience 
% 
% Todo:
% 
% Done:
% - Chris to help  motivate this paragraph with some *literature on 
% empirical motivation for why this is important. (GP, 2002)?
% 

An interesting implication is that the consumption function exists (and is not degenerate) even when the (exogenous) growth rate of income exceeds the (exogenous) interest rate.
Many economic models impose an interest rate greater than the growth rate because if that condition does not hold then the risk-neutral present discounted value of future income is infinite.
The presence of the precautionary motive short-circuits this logic, and implies that even if \textit{in a risk-neutral sense} human wealth is unbounded as the planning horizon receeds, the limiting solution is not $c = \infty$.
Using this insight, the final section shows that for a consumer to have a non-degenerate value function in the limit as the planning horizon recedes, \hyperlink{APF}{absolute patience} cannot exceed \textit{both} market returns and the growth rate of income.
Growth impatience must hold when return impatience fails and vice-versa.
When both growth impatience and return impatience fail, the limiting consumption function is either $\cFunc(\mNrm)=0$ or $\cFunc(\mNrm)=\infty$ for all $\mNrm$.

%AS to AS as a note: The presence of the precautionary motive short-circuits this logic -> because of the natural borrowing constraint. 

\vspace{-1em}

\hypertarget{cfLiterature}{}
\hypertarget{DiffFromLit}{} 
\paragraph{Relationship to Literature}

% Part 8: Literature: Lit of BSS
% - State key non-theory history of BSS
% - State related concepts literature (prec. saving)
% Todo:
% - Chris to amend this to make it more accurate. 

Buffer stock saving behaviour was recognized by \cite{friedmanATheory} and formally introduced by \cite{carrollBSLCPIH} to account for consumption and income patterns in the data.
The concept is closely linked to precautionary saving \citep{zeldesStochastic, deatonLiqConstr, kimballPrecautionarySaving} and the literature \citep{ckConcavity, lightPrecautionarySaving, chkLiqConstr, maSavingRateRich} now provides theoretical  results on how risk affects the consumption function in the presence of the precautionary motive.
The class of models we use to study buffer stock saving, income fluctuation problems, are now pervasive, though the foundational contributions include \cite{bewleyPIH}, \cite{imrohorogluBusinessCycles}, \cite{zeldesStochastic}, \cite{deatonLiqConstr}, \cite{huggettRiskFreeRate} and \cite{aiyagari:ge}.
Amongst this literature \cite{zeldesStochastic} was the first to calibrate a quantitatively plausible example of permanent and transitory shocks and argue that the natural borrowing constraint was a quantitatively plausible alternative to `artificial' or `ad hoc' borrowing constraints.\footnote{The same (numerical) point applies for infinite horizon models (calibrated to actual empirical data on household income dynamics); cf.~\cite{carrollBrookings}.} The natural borrowing limit was also described by \cite{aiyagari:ge}, but implications for existence not discussed.\footnote{Income fluctuation problems do not require existence of stable buffer stock targets, though such points will often exist.}


% Part 10: Literature: Lit on income fluctution technical results 
% 
% Todo: - \cite{rabaultBorrowing} do they actually show existence?AS TO AS check this. 

On the technical front, traditional Bellman iteration approaches to showing existence rely on bounded pay-offs~\citep{slpMethods}.\footnote{The CRRA utility function does not satisfy Bewley's assumption that $\uFunc(0)$ is well-defined, or that $\uP(0)$ is well-defined and finite.
Our framework differs from Schectman and Escudero~\citeyearpar{seIncFluct} in that they impose liquidity constraints and positive minimum income.
It differs from Deaton~\citeyearpar{deatonLiqConstr} because liquidity constraints are absent; there are separate transitory and permanent shocks (\textit{a la}~\cite{muthOptimal}); and the transitory shocks here can occasionally cause income to reach zero.
Similar restrictions are made in the well known papers by Scheinkman and Weiss~\citeyearpar{scheinkman&weiss:borrowing}, Clarida~\citep{claridaErgodic}, and others~\cite{cwcUnderUncert}.
For a related continuous-$t$ model, see~\cite{tocheUrisk}.
\cite{asHomogeneous} relaxed the bounds on the return function, but they address only the deterministic case with compact valued action sets.
\cite{mnUnique} assume a framework with compact action sets, and real-valued pay-offs, which cannot handle CRRA utility unbounded below.
See~\cite{yaoNote} for a detailed discussion of the reasons the existing literature up through~\cite{mnUnique} cannot handle the problem described here.
\@~\cite{mvExistence} provide a correction to~\cite{rrExistence}, but only addresses the deterministic case.} Overcoming these restrictions to allow unbounded pay-offs, the literature on the one hand emphasized time iteration operators defined by Euler equations ~\citep{deatonLiqConstr, lsIncFluct, mstIncFluct} and, on the other hand, transformations of the Bellman equation  \citep{maUnboundedDP, rinconZapatero2024}.
The results by \cite{mstIncFluct,maUnboundedDP} are the most general we are aware of that tackle income fluctuation problems, and can be specialized to show existence in a model with the rate of return and discount factor shock structure arising from permanent and transitory shocks (once the model is normalized).
However, the cited approach impose an artificial liquidity constraint, which effectively bound utility from below, and thus cannot be applied here.
While \cite{rinconZapatero2024} allows utility to be unbounded below, consuming nothing cannot yield a value of $-\infty$, which again would require us to impose an artificial liquidity constraint.


Moreover, our  growth and return patience concern economic mechanisms (rather than general assumptions) that arise in the presence of permanent income uncertainty and growth.
To the best of our knowledge, these economic mechanisms have not been explored elsewhere.


% Part 11: Literature: Lit on more abstract technical methods 
% 

Our approach to constructing the weighted-norm space of value functions uses results on unbounded dynamic programming by  \citep{jboydWeighted}.\footnote{\cite{asHomogeneous} showed how the approach could be used to address the homogeneous case (of which CRRA is an example) in a deterministic framework; later,~\cite{duranDiscounting} showed how to extend the~\cite{jboydWeighted} approach to the stochastic case.
See also exposition by \cite{stachurski2022}, Ch.
12.} However, our use of marginal propensities to consume to construct per-period bounds on the Bellman operator are novel.
In a more abstract setting, our proofs address issues of compactness and continuity similar to that tackled by ~\cite{Feinberg2012} and  \cite{Shanker2017a}.
In contrast to the abstract methods, the proofs in this paper are directly applied to an income fluctuation problem and have the advantage that they employ standard concepts, such as Bellman iteration, that are straightforward to verify in practice.

% Part 12: Literature: Lit onstochastic stability 
% 
% Todo: Akshay to add in some refs to Toda and others. AS to AS: add in distribution tails stuff here. 

Finally, our discussion on aggregate growth rates builds on \cite{szeidlInvariant} and \cite{harmenbergInvariant} who give results on the existence and convergence of stationary wealth distributions for the model presented here.
\cite{mstIncFluct} also give results on stationarity, under the restrictions mentioned above.
While conditions for stationarity resemble growth impatience and strong growth impatience our objective is to establish existence of stable buffer stock targets, which have empirical relevance, not prove stochastic stability.

% AS to AS: empirical relevance for what? AS to CDC this one quickly. 

\compilingasstandalone{
  \econarkmultibib{Introduction}
}

\end{document}

% TeX-master-file: t
% eval: (setq TeX-command-list  (assq-delete-all (car (assoc "BibTeX" TeX-command-list)) TeX-command-list))
% eval: (setq TeX-command-list  (assq-delete-all (car (assoc "Biber"  TeX-command-list)) TeX-command-list))
% eval: (add-to-list 'TeX-command-list '("BibTeX" "bibtex %s" TeX-run-BibTeX nil t                                                                              :help "Run BibTeX") t)
% eval: (add-to-list 'TeX-command-list '("BibTeX" "bibtex %s" TeX-run-BibTeX nil (plain-tex-mode latex-mode doctex-mode ams-tex-mode texinfo-mode context-mode) :help "Run BibTeX") t)
% TeX-PDF-mode: t
% TeX-file-line-error: t
% TeX-debug-warnings: t
