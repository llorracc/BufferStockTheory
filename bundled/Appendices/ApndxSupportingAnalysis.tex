% -*- mode: LaTeX; TeX-PDF-mode: t; -*-
\input{@resources/tex-add-search-paths}  % allow latex to find custom stuff
% this file is in the root directory so . is the path to the root 
\providecommand{\econtexRoot}{}\renewcommand{\econtexRoot}{.}

\documentclass[\econtexRoot/BufferStockTheory]{subfiles}

\newcommand{\subname}{ApndxSupportingAnalysis}
%\providecommand{\ApndxDir}{}\renewcommand{\ApndxDir}{\econtexRoot/Appendices}
\providecommand{\EqDir}{}\renewcommand{\EqDir}{\econtexRoot/Equations}
\providecommand{\TableDir}{}\renewcommand{\TableDir}{\econtexRoot/Tables}
\providecommand{\FigDir}{}\renewcommand{\FigDir}{\econtexRoot/Figures}
\providecommand{\LaTeXInputs}{}\renewcommand{\LaTeXInputs}{\econtexRoot/@resources/texlive/texmf-local/tex/latex}
\providecommand{\LaTeXGenerated}{}\renewcommand{\LaTeXGenerated}{\econtexRoot} % not worth trying to put generated files in a subdir
\providecommand{\ResourcesDir}{}\renewcommand{\ResourcesDir}{\econtexRoot/@resources}
\providecommand{\LtxDir}{}\renewcommand{\LtxDir}{}
 % get directory macros
\usepackage{econark-ifsubfile}        % allow conditional execution of code
\usepackage{econark-xrsetup}          % Xternal crossReferences (from main document)

\compilingasstandalone{
  \xrsetup{\econtexRoot/\texname}
}
\begin{document}

\section{Supporting Standard Results in Real Analysis}\label{sec:realanalysis}

%\notinsubfile{\pagebreak\bibliographystyle{plainnat}\econarkmultibib{\texname}}

\begin{proposition}\label{prop:xnconvgf}
Let $\fFunc: \Reals_{++} \to \Reals_{+}$ be a continuous function. Consider sequences $x^{n}$ in $\Reals_{++}$ and $\fFunc^n(x^{n})$ in $\Reals_{+}$. If $\fFunc^{n}(x^{n}) \to \fFunc(x)$ as $n \to \infty$, then $x^{n} \to x$ as $n \to \infty$.
\end{proposition}

\begin{proof}
Given that $\fFunc$ is continuous at $x$ (with $x \in \Reals_{++}$), for every $\epsilon > 0$, there exists a $\delta > 0$ such that for all $y$ in $\Reals_{++}$ with $\vert y - x\vert < \delta$, we have $\vert \fFunc(y) - \fFunc(x)\vert  < \epsilon$.

Given $\fFunc^{n}(x^{n}) \to \fFunc(x)$, for the above $\epsilon$, there exists an $N$ such that for all $n > N$, $\vert \fFunc^{n}(x^{n}) - \fFunc(x)\vert < \epsilon$.

Assume for the sake of contradiction that $x^{n}$ doesn't converge to $x$.
This implies that there exists a $\delta > 0$ such that for infinitely many terms of the sequence $x^{n}$, $\vert x^{n} - x\vert  \geq \delta$.

By the continuity of $\fFunc$ at $x$, if $\vert x^{n} - x\vert  \geq \delta$ for infinitely many $n$, then $\vert \fFunc^{n}(x^{n}) - \fFunc(x)\vert  \geq \epsilon$ for those $n$, contradicting our assumption that $\fFunc^{n}(x^{n}) \to \fFunc(x)$.

Therefore, our assumption for contradiction is false, and it follows that $x^{n} \to x$ as $n \to \infty$.
\end{proof}

\begin{fact}
Let $\gFunc: X \to \Reals_{+}$ be a continuous function, where $X \subseteq \Reals^{n}$ is an open convex set. Define the weighted supremum norm $\Vert\cdot \Vert_{\gFunc}$ of a real-valued function $\fFunc: X \to \Reals$ by
\begin{equation}
\Vert \fFunc \Vert_\gFunc := \sup_{x \in X} \frac{\vert \fFunc(x)\vert }{\gFunc(x)}.
\end{equation}
If $\lim_{n \to \infty} \Vert \fFunc_n - \fFunc^{\star} \Vert_\gFunc = 0$, $\fFunc_n$ converges to $\fFunc^{\star}$ uniformly on compact sets.
\end{fact}

\begin{proof}
Let $\tilde{X}$ be an arbitrary compact subset of $X$.
Since $\tilde{X}$ is compact, there exists a positive lower bound for $\gFunc$ on this subset, denoted as 
\begin{equation}
\bar{\gFunc} = \min_{x \in \tilde{X}} \gFunc(x) > 0.
\end{equation}
Hence, on $\tilde{X}$, if $\lim_{n \to \infty} \Vert \fFunc_n - \fFunc^{\star} \Vert_\gFunc = 0$, then $\lim_{n \to \infty} \Vert \fFunc_n - \fFunc^{\star} \Vert_\infty = 0$ on $\tilde{X}$, where $\Vert \cdot \Vert_\infty$ denotes the supremum norm.

% CDC-to-AS: There are two versions of this proof.  Which should we keep?

Now, let $K$ be a compact subset of $X$.
Given the continuity of $\gFunc$, there exists a positive maximum value for $\gFunc$ on $K$, denoted as $M_K$.
Then, we have
\begin{equation}
\sup_{x \in K} \vert \fFunc_n(x) - \fFunc(x)\vert  \leq M_K \sup_{x \in K} \frac{\vert \fFunc_n(x) - \fFunc(x)\vert }{\gFunc(x)} \leq M_K \sup_{x \in X} \frac{\vert \fFunc_n(x) - \fFunc(x)\vert }{\gFunc(x)}.
\end{equation}
Thus, $\lim_{n \to \infty} \Vert \fFunc_n - \fFunc \Vert_  = 0$ implies that $\fFunc_n$ converges uniformly to $\fFunc$ on the compact set $K$.
It's also worth noting that the convexity and openness of $X$ aren't strictly necessary for this argument.
\end{proof}

\begin{fact}\label{fact:compactnt}
Let $\{\fFunc_n\}$ be a sequence of continuous functions defined on a subset of the real line and converging uniformly to a function $\fFunc$ on compact sets. If $\{x_n\}$ is a convergent sequence of real numbers with limit $x$, then $\fFunc_n(x_n)$ converges to $\fFunc(x)$.
\end{fact}

\begin{proof}
Let $\tilde{X}$ be an arbitrary compact subset of $X$.
Since $\tilde{X}$ is compact, there exists a positive lower bound for $\gFunc$ on this subset, denoted as 
\begin{equation}
\bar{\gFunc} = \min_{x \in \tilde{X}} \gFunc(x) > 0.
\end{equation}
Hence, on $\tilde{X}$, if $\lim_{n \to \infty} \Vert \fFunc_n - \fFunc^{\star} \Vert_\gFunc = 0$, then $\lim_{n \to \infty} \Vert \fFunc_n - \fFunc^{\star} \Vert_\infty = 0$ on $\tilde{X}$, where $\Vert \cdot \Vert_\infty$ denotes the supremum norm.


Now, let $K$ be a compact subset of $X$.
Given the continuity of $\gFunc$, there exists a positive maximum value for $\gFunc$ on $K$, denoted as $M_K$.
Then, we have
\begin{equation}
\sup_{x \in K} \vert \fFunc_n(x) - \fFunc(x)\vert  \leq M_K \sup_{x \in K} \frac{\vert \fFunc_n(x) - \fFunc(x)\vert }{\gFunc(x)} \leq M_K \sup_{x \in X} \frac{\vert \fFunc_n(x) - \fFunc(x)\vert }{\gFunc(x)}.
\end{equation}
Thus, $\lim_{n \to \infty} \Vert \fFunc_n - \fFunc \Vert_  = 0$ implies that $\fFunc_n$ converges uniformly to $\fFunc$ on the compact set $K$.
It's also worth noting that the convexity and openness of $X$ aren't strictly necessary for this argument.
\end{proof}

\begin{fact}\label{fact:compactnts}
Let $\{\fFunc_n\}$ be a sequence of continuous functions defined on a subset of the real line and converging uniformly to a function $\fFunc$ on compact sets. If $\{x_n\}$ is a convergent sequence of real numbers with limit $x$, then $\fFunc_n(x_n)$ converges to $\fFunc(x)$.
\end{fact}

\begin{proof}
Since $x_n$ converges to $x$, the sequence $\{x_n\}$ is bounded.
Therefore, there exists a compact set $K$ (specifically, a closed interval in the real line) that contains all the $x_n$ as well as $x$.

Given the uniform convergence of $\fFunc_n$ to $\fFunc$ on $K$, for every $\epsilon > 0$, there exists an $N$ such that for all $n \geq N$ and for all $y \in K$, we have
\[ \vert \fFunc_n(y) - \fFunc(y)\vert  < \epsilon.
\]
In particular, for $y = x_n$, we have
\[ \vert \fFunc_n(x_n) - \fFunc(x_n)\vert  < \epsilon \]
for all $n \geq N$.

Now, each $\fFunc_n$ being continuous and $x_n$ converging to $x$ implies that $\fFunc(x_n)$ converges to $\fFunc(x)$.
Thus, for sufficiently large $n$, $\fFunc(x_n)$ can be made arbitrarily close to $\fFunc(x)$.

Combining the two inequalities and taking $n$ large enough, we deduce that $\vert \fFunc_n(x_n) - \fFunc(x)\vert$ can be made smaller than any given $\epsilon$.
Hence, $\fFunc_n(x_n)$ converges to $\fFunc(x)$.
\end{proof}



\end{document}
\endinput

% Local Variables:
% eval: (setq TeX-command-list  (remove '("Biber" "biber %s" TeX-run-Biber nil  (plain-tex-mode latex-mode doctex-mode ams-tex-mode texinfo-mode)  :help "Run Biber") TeX-command-list))
% eval: (setq TeX-command-list  (remove '("Biber" "biber %s" TeX-run-Biber nil  t  :help "Run Biber") TeX-command-list))
% eval: (setq TeX-command-list  (remove '("BibTeX" "%(bibtex) %s"    TeX-run-BibTeX nil t :help "Run BibTeX") TeX-command-list))
% eval: (setq TeX-command-list  (remove '("BibTeX" "bibtex %s"    TeX-run-BibTeX nil t :help "Run BibTeX") TeX-command-list))
% tex-bibtex-command: "bibtex.*"
% TeX-PDF-mode: t
% TeX-file-line-error: t
% TeX-debug-warnings: t
% LaTeX-command-style: (("" "%(PDF)%(latex) %(file-line-error) %(extraopts) -output-directory=. %S%(PDFout)"))
% TeX-source-correlate-mode: t
% TeX-parse-self: t
% eval: (cond ((string-equal system-type "darwin") (progn (setq TeX-view-program-list '(("Skim" "/Applications/Skim.app/Contents/SharedSupport/displayline -b %n %o %b"))))))
% eval: (cond ((string-equal system-type "gnu/linux") (progn (setq TeX-view-program-list '(("Evince" "evince --page-index=%(outpage).%o"))))))
% eval: (cond ((string-equal system-type "gnu/linux") (progn (setq TeX-view-program-selection '((output-pdf "Evince"))))))
% TeX-parse-all-errors: t
% End:
