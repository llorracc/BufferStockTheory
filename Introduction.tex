\input{@resources/tex-add-search-paths}  % allow latex to find custom stuff
% -*- mode: LaTeX; TeX-PDF-mode: t; -*-
% this file is in the root directory so . is the path to the root 
\providecommand{\econtexRoot}{}\renewcommand{\econtexRoot}{.}

\documentclass[BufferStockTheory]{subfiles}


\newcommand{\subname}{Introduction}   % Needed for generic bib to work
\providecommand{\ApndxDir}{}\renewcommand{\ApndxDir}{\econtexRoot/Appendices}
\providecommand{\EqDir}{}\renewcommand{\EqDir}{\econtexRoot/Equations}
\providecommand{\TableDir}{}\renewcommand{\TableDir}{\econtexRoot/Tables}
\providecommand{\FigDir}{}\renewcommand{\FigDir}{\econtexRoot/Figures}
\providecommand{\LaTeXInputs}{}\renewcommand{\LaTeXInputs}{\econtexRoot/@resources/texlive/texmf-local/tex/latex}
\providecommand{\LaTeXGenerated}{}\renewcommand{\LaTeXGenerated}{\econtexRoot} % not worth trying to put generated files in a subdir
\providecommand{\ResourcesDir}{}\renewcommand{\ResourcesDir}{\econtexRoot/@resources}
\providecommand{\LtxDir}{}\renewcommand{\LtxDir}{}
 % get directory macros
\usepackage{econark-ifsubfile}
\usepackage{econark-xrsetup}
\externaldocument{\LaTeXGenerated/BufferStockTheory}

% set up identity of the external source of refs and bibs
\xrsetup{\econtexRoot/\texname}

\newcommand{\thankstext}{
  The paper's results \href{https://\owner.github.io/nbreproduce}{can be automatically reproduced} using the {\ARKurl} toolkit, which can be cited per our references (\cite{carroll_et_al-proc-scipy-2018}); for reference to the toolkit itself see \href{https://econ-ark.org/acknowledging}{Acknowledging Econ-ARK}.
  Thanks to the \href{https://consumerfinance.gov}{Consumer Financial Protection Bureau} for funding the original creation of the {\ARKurl} toolkit; and to the \href{https://sloan.org}{Sloan Foundation} for funding Econ-ARK's \href{https://sloan.org/grant-detail/8071}{extensive åfurther development} that brought it to the point where it could be used for this project.
  The toolkit can be cited with its digital object identifier, \href{https://doi.org/10.5281/zenodo.1001067}{10.5281/zenodo.1001067}, as is done in the paper's own references as \cite{carroll_et_al-proc-scipy-2018}.
  Thanks to Will Du, James Feigenbaum, Joseph Kaboski, Miles Kimball, Qingyin Ma, Misuzu Otsuka, Damiano Sandri, John Stachurski, Adam Szeidl, Alexis Akira Toda, Metin Uyanik, Mateo Vel\'asquez-Giraldo, Weifeng Wu,  Jiaxiong Yao, and Xudong Zheng for comments on earlier versions of this paper, John Boyd for help in applying his weighted contraction mapping theorem, Ryoji  Hiraguchi for extraordinary mathematical insight that improved the  paper greatly, David Zervos for early guidance to the literature, and participants in a seminar at the Johns Hopkins University, a presentation at the 2009 meetings of the Society of Economic Dynamics for their insights, and at a presentation at the Australian National University.}

% \newcommand{\inframe}{https://\owner.github.io/BufferStockTheory/BufferStockTheory3.html}

% Embed metadata
\hypersetup{pdfauthor={Christopher D Carroll <ccarroll@jhu.edu>},
  pdftitle={Theoretical Foundations of Buffer Stock Saving},
  pdfkeywords={Precautionary saving, buffer-stock saving, consumption, marginal propensity to consume, permanent income hypothesis, income fluctuation problem},
  pdfnewwindow=true,
  pdfcreator = {ccarroll@jhu.edu}
}

\begin{document}

% Attempted to make all lines used for Web version contain {Web} (or version with only single curly brace at end) so can be removed with sed
\ifthenelse{\boolean{Web}}{    % {Web}
  \renewcommand{\versn}{Web}     % Too hard to figure out passing -output-directory through make4ht through htlatex, so web version is compiled with junk files in main directory
  \renewcommand{\rootFromOut}{.} % {Web}
}{}  % {Web}


% Tiny info header at top tracks git commit

\title{Theoretical Foundations of \\ Buffer Stock Saving}

%\author{Christopher D Carroll\authNum ~and Akshay Shanker\authNum}
\author{Christopher D Carroll         ~and Akshay Shanker}

\keywords{Precautionary saving, buffer stock saving, marginal propensity to consume, permanent income hypothesis, income fluctuation problem}

\jelclass{D81, D91, E21 \par
  \href{https://econ-ark.org}{\includegraphics{@resources/econ-ark/PoweredByEconARK}}
}

\date{\today}
% \renewcommand{\forcedate}{April 24, 2023}\date{\forcedate}

\maketitle
\hypertarget{abstract}{}
\begin{abstract}
  This paper builds foundations for rigorous and intuitive understanding of `buffer stock' saving behaviour that can emerge in \cite{bewleyPIH}-like economies.
  After describing conditions under which a consumption function exists, the paper articulates stricter `Growth Impatience' conditions sufficient to guarantee the existence of a buffer-stock target --- either at the population level, or for individual consumers.
  Together, these analytical results, along with the inlcuded numerical illustrations, constitute a comprehensive toolkit for understanding buffer stock saving.
\end{abstract}

% Various resources
\hypertarget{links}{}

\begin{footnotesize}
  \parbox{0.9\textwidth}{
    \begin{center}
      \begin{tabbing}
        \texttt{~~~\REMARK:~} \= \= \texttt{\url{https://econ-ark.org/materials/bufferstocktheory}} \\ 
        \texttt{Dashboard:~} \> \> \texttt{\href{https://econ-ark.org/materials/bufferstocktheory}{Click `Launch Dashboard' Button}} \\
        \texttt{~~~~~html:~} \> \> \texttt{\href{https://\owner.github.io/BufferStockTheory/}{https://\owner.github.io/BufferStockTheory/}} \\ % Owner is defined in Resources/owner.tex
        \texttt{~~~~~~PDF:~} \> \> \texttt{\href{https://github.com/\owner/BufferStockTheory/blob/master/BufferStockTheory.pdf}{BufferStockTheory.pdf}} \\ 
        \texttt{~~~Slides:~} \> \> \texttt{\href{https://github.com/\owner/BufferStockTheory/blob/master/LaTeX/BufferStockTheory-Slides.pdf}{BufferStockTheory-Slides.pdf}} \\
        % \texttt{~Appendix:~} \> \> \texttt{\url{https://\owner.github.io/BufferStockTheory/BufferStockTheory3.html#Appendices}}    \\
        \texttt{~~~GitHub:~} \> \> \texttt{\href{https://github.com/\owner/BufferStockTheory}{https://github.com/\owner/BufferStockTheory}} \\
      \end{tabbing}
    \end{center}
    % The \href{https://econ-ark.org/materials/bufferstocktheory\?dashboard}{dashboard} lets users see consequences of alternative parameters in an interactive framework.
  } % end \parbox{\textwidth}
\end{footnotesize}

% \begin{authorsinfo}
%   \name{Contact: \href{mailto:ccarroll@jhu.edu}{\texttt{ccarroll@jhu.edu}}, Department of Economics, 590 Wyman Hall, Johns Hopkins University, Baltimore, MD 21218, \url{https://www.econ2.jhu.edu/people/ccarroll}, and National Bureau of Economic Research.}\name{Contact: \href{mailto:akshay.shanker@unsw.edu.au}{\texttt{akshay.shanker@unsw.edu.au}}, Department of Economics, University of New South Wales, Sydney, Australia and Australian Research Council (ARC) Center for Excellence in Population and Ageing Research (CEPAR).}
% \end{authorsinfo}

\pagenumbering{gobble} % Prevent numbering for pages including the TOC and title page
\hypersetup{pageanchor=false}  % https://tex.stackexchange.com/questions/18924/pdftex-warning-ext4-destination-with-the-same-identifier-nam-epage-1-has
% \ifthenelse{\boolean{Web}}{ % Web 
% }{ % Web 
  \begin{minipage}{0.9\textwidth} 
    \footnotesize The paper's results \href{https://\owner.github.io/nbreproduce}{can be automatically reproduced} using the {\ARKurl} toolkit by executing the \href{https://econ-ark.org/materials/bufferstocktheory}{notebook}; for reference to the toolkit itself see \href{https://econ-ark.org/acknowledging}{Acknowleding Econ-ARK}.  Thanks to the \href{https://consumerfinance.gov}{Consumer Financial Protection Bureau} for funding the original creation of the {\ARKurl} toolkit; and to the \href{https://sloan.org}{Sloan Foundation} for funding Econ-ARK's \href{https://sloan.org/grant-detail/8071}{extensive further development} that brought it to the point where it could be used for this project.  The toolkit can be cited with its digital object identifier, \href{https://doi.org/10.5281/zenodo.1001067}{https://doi.org/10.5281/zenodo.1001067}, as is done in the paper's own references as \cite{carroll_et_al-proc-scipy-2018}.  Thanks to Will Du, James Feigenbaum, Joseph Kaboski, Miles Kimball, Qingyin Ma, Misuzu Otsuka, Damiano Sandri, John Stachurski, David Stern, Adam Szeidl, Alexis Akira Toda, Metin Uyanik, Mateo Vel\'asquez-Giraldo, Weifeng Wu,  Jiaxiong Yao, and Xudong Zheng for comments on earlier versions of this paper, John Boyd for help in applying his weighted contraction mapping theorem, Ryoji  Hiraguchi for extraordinary mathematical insight that improved the  paper greatly, David Zervos for early guidance to the literature, and participants in a seminar at the Johns Hopkins University, a presentation at the 2009 meetings of the Society of Economic Dynamics for their insights, and at a presentation at the Australian National University. Shanker gratefully acknowledges research support from the Australian Research Council (ARC LP190100732) and ARC Centre of Excellence in Population Ageing Research (CE17010005). 
  \end{minipage} 

  \titlepagefinish\pagebreak
  \let\LaTeXStandardContentsName\contentsname
  \renewcommand{\contentsname}{}
  \tableofcontents
  \pagebreak
  \typeout{after \\tableofcontents}
  \medskip\medskip
  \begin{minipage}{0.9\textwidth}
    \listoffigures 
  \end{minipage}

  \medskip\medskip
  \begin{minipage}{0.9\textwidth}
    \listoftables
  \end{minipage}
%} % Web

\hypersetup{pageanchor=true}  % https://tex.stackexchange.com/questions/18924/pdftex-warning-ext4-destination-with-the-same-identifier-nam-epage-1-has

\pagebreak
\hypertarget{Introduction}{}
\section{Introduction}\label{sec:intro}
\setcounter{page}{0}\pagenumbering{arabic}

The precautionary motive to save springs from the fact that extra resources improve a consumer's ability to buffer spending against shocks.
A consumer who, in the absence of shocks, would be impatient enough to plan to spend down their resources, will (when shocks are present) experience an intensifying precautionary motive as their buffering capacity shrinks.
The result of this competition between impatience and `prudence' \citep{kimball:smallandlarge} has been described, starting with \cite{deatonLiqConstr}, as `buffer stock saving,' with a `target' defined\footnote{In \cite{carroll:brookings}.} as the point where the precautionary motive to accumulate becomes exactly strong enough to counter the impatience motive to decumulate.

The logic of buffer stock saving underpins key findings in heterogeneous-agent (HA) macroeconomics.
For example, it can explain why, during the Great Recession, middle-class consumers cut spending more than the poor or the rich~\citep{kmpHandbook}.
Buffer stock saving also can explain why consumption growth tracks income growth over much of the life cycle,\footnote{\cite{carrollBSLCPIH}, \cite{gpLifeCycle}} rather than being determined solely by preferences and interest rates as Irving~\cite{fisherInterestTheory} proposed.

Buffer stock saving models are neither a subset nor a superset of the closely-related class of \cite{bewleyPIH} models (or, more generically, \cite{schectman:fluctuation} `income fluctuation' problems; we will use the terms `Bewley model' and `income fluctuation problem' interchangeably).
That is, not all Bewley models with fluctuating income exhibit buffer stock saving, and not all models that exhibit buffer stock saving satisfy the mathematical assumptions that guarantee boundedness of marginal marginal utility imposed by Schectman or Bewley (and inherited by almost all of the subsequent literature through to the recent contributions of \cite{mstIncFluct,maUnboundedDP}). %

The purpose of this paper is to provide a comprehensive statement and explanation of the conditions under which buffer stock saving arises in a class of problems broader in important respects than Bewley models. Specifically, we consider the problem of an agent who is subject to a Friedman-Muth(-Zeldes) income process incorporating permanent shocks to noncapital income~\citep{friedmanATheory, muthOptimal, zeldesStochastic} in addition to the transitory shocks traditionally examined in the income fluctuation literature,\footnote{It is our view that the principal reason much of the literature has incorporated extremely `persistent' but not completely permanent shocks is that the theoretical foundations for the case with permanent shocks have not previously been available.}
and who does not face an `artificial' borrowing constraint (a constraint that prohibits borrowing even when the loan could certainly be repaid).

In the course of proving our main theoretical results, we define a variety of alternative measures of `patience' -- an intuitive term that nonetheless has had multiple interpretations in the literature. Different measures of impatience guarantee two kinds of results: The existence of a nondegenerate limiting consumption function, and the existence of a buffer stock `target.'

\paragraph{Patience Requirements for Nondegeneracy}
% define (non)degeneracy
We will define the `limiting' consumption function as the limit of the sequence of consumption rules constructed by iterating backward from a terminal period $T$, and we will say that this limiting function is `nondegenerate' if it exists, is real-valued, and is strictly positive for every reachable circumstance the consumer could be in.


%For example, we will define a perfect foresight consumer as exhibiting `growth impatience' if they would always like to spend more than their current income.
%This can occur even if their time preference factor is $\DiscFac=1$.
%The key insight is that the existence of non-degenerate limiting solutions and buffer stock targets depends on the relation between the consumer's preferences (both time preference and risk aversion) and the exogenous features of their environment (the income process and the interest factor $\Rfree$).

%in relation to the rate of return ($\Rfree$) and stochastic rate of permanent income growth ($\PermGroFacRnd$).
%However, while the discount factor, $\DiscFac$, has been known to represent the consumer's `pure' rate of time preference, reflecting the relative weight of utility over time, `patience' lacks a similarly clear understanding in the literature.

%To grasp the notion of consumer patience, consider a perfect foresight scenario with no borrowing constraints.
%In this setting, a consumer optimally balances current wealth between present and future consumption, resulting in a positive marginal propensity to consume.
%For example, in the unconstrained perfect foresight model the growth factor for consumption is $\cLvl_{t+1}/\cLvl_{t} = (\Rfree \DiscFac)^{1/\CRRA}$. 

\newcommand{\APFacRaw}{(\Rfree\DiscFac)^{1/\CRRA}}
% When is it same as Bewley?
When an artificial borrowing constraint is imposed and noncapital income is stationary -- it is subject to no permanent shocks and exhibits no long-term growth %($\PermGroFac = 1$ in our notation)
-- our problem coincides with a standard `income fluctuation problem.' But our new proof methods allow us also to solve models where permanent income is unbounded above and below and there is no artificial constraint.

% In PF model:
% - FHWC
% - RIC

In the unconstrained perfect foresight version of our problem, one type of degeneracy arises if permanent income perpetually grows faster than the rate at which it is discounted: With no limiting upper bound to the PDV of future income and no borrowing constraint, there is no upper bound to limiting consumption. Imposition of a `Finite Human Wealth' condition ($\PermGroFac < \Rfree$) is required to prevent this. Another type of degeneracy arises if preferences fail the `return impatience' condition: $\APFacRaw/\Rfree < 1$.  Without return impatience, the limiting consumption function is zero everywhere.

% Add FM uncertainty?
% - WRIC
% - NBS eliminates need for FHWC
% - 
Intuition would suggest that, by activating the precautionary saving motive, introduction of the Friedman-Muth stochastic income process might necessitate a stronger degree of impatience to avoid degeneracy. In fact, we show that the required impatience condition is \textit{weaker}, for reasons that follow from the lack of an artificial borrowing constraint and the presence of a `natural' borrowing constraint,\footnote{The `natural' constraint arises as a consequence of the budget constraint and the CRRA utility function which implies that the utility of consuming zero is negative infinity.  Its implications were first explored by \cite{zeldesStochastic} in a life cycle context. \cite{carroll:brookings} analyzed the infinite horizon case, and \cite{aiyagari:ge} coined the term `natural borrowing constraint.'} (which has the additional effect of eliminating the need to impose the Finite Human Wealth condition). We further demonstrate that the artificial constraint emerges as the limit, as a certain parameter goes to zero, of the natural constraint. This provides an intuitive conceptual bridge between the two.

In the existing literature going all the way back to \cite{fisherInterestTheory},\footnote{\cite{fisherInterestTheory} in Ch. IV, Section 3 states "I shall treat the two terms (impatience and time preference) as synonymous".}, time preference `$\DiscFac$' and patience have often been treated as synonymous. %

We show that, in the presence of nonstationary permanent income, the corresponding generalized mathematical steps yield a \hyperlink{FVAC}{`finite value of autarky'} condition that incorporates characteristics of the growth process. The fact that the generalized condition involves terms other than $\DiscFac$ undermines the temptation to identify `patience' solely with the pure time preference factor.  We therefore propose that henceforth the literature should deprecate use of the term `patience' unadorned with any adjective identifying precisely which \textit{kind} of patience is under consideration. 

The final section shows that for a consumer to have a non-degenerate value function in the limit as the planning horizon recedes, \hyperlink{APF}{absolute patience} cannot exceed \textit{both} the market return factor $\Rfree$ and the income growth factor $\PermGroFac.$
(Growth impatience must hold if return impatience fails and vice-versa.)
When both growth impatience and return impatience fail, the limiting consumption function is either $\cFunc=0$ or $\cFunc=\infty$. 

\paragraph{Patience Requirements and the Existence of Buffer Stock Targets}%Turning to our results on the buffer stock target, t

Once we have established the existence of a non-degenerate solution, the second (and more important) main result of the paper is to identify conditions under which buffer stock `targets' exist, for individual consumers or in the aggregate.

As noted by \cite{szeidlInvariant} and \cite{maUnboundedDP}, the impatience condition $(\Rfree \DiscFac) < 1$ is commonly imposed in Bewley models to guarantee existence and stability of the stochastic distributions of wealth and other variables.  % We show that $(\Rfree\DiscFac)<1$ is neither necessary nor sufficient to ensure the existence of a non-degenerate limiting consumption function. %AS to look into. 

The appropriate definition of a buffer stock `target' turns out to depend on whether we are interested in the microeconomic behavior of individual consumers, or the aggregate behavior of the entire population of consumers.
The requirement for the existence of an individual target is \hyperlink{GICMod}{`strong growth impatience,'} ($\Ex[\APFacRaw/\PermGroFacRnd] < 1$) which prevents the household's `normalized market resources' $\mNrm = \mLvl/\pLvl$ from growing without bound.
Specifically, strong growth impatience guarantees that at some large-enough value $\mNrm = \acute{\mNrm}$ it must be the case that the expectation of next period's $\mNrm$ is less than this period's: (if $\mNrm_{t} > \acute{m}$, then $\Ex_{t}[\mNrm_{t+1}] < \mNrm_{t}]$). 
This turns out to guarantee that normalized market resources eventually revert back toward a target.

A weaker requirement, \hyperlink{GIC}{`growth impatience,'} ensures the existence of an aggregate buffer stock target even when individual target ratios are unbounded.
Growth impatience requires the ratio of \hyperlink{APFacDefn}{absolute patience} to the \textit{expected} growth factor of permanent income to be less than one: $\APFacRaw/\PermGroFac < 1$.


The trick to understanding how there can be an aggregate target even when there is no individual target is to realize that one reason that $\mLvl/\pLvl$ can grow is that individuals can have negative shocks to $\pLvl$.  But the people whose ratio grows because their $\pLvl$ shrinks by definition account for a smaller portion of the \textit{level} of aggregate permanent income. That is, even as their $\mNrm$ rises they become smaller contributors to the aggregate economy.  


Thus in the aggregate, even with a fixed interest rate that differs from the time preference rate, a small open economy populated by buffer stock consumers has a balanced growth path in which growth rates of consumption, income, and market resources match the exogenous growth rate of aggregate permanent income (equivalent, here, to productivity growth).
In the terms of~\cite{schmitt2003closing}, buffer stock saving is an appealing method of `closing' a small open economy model, because it requires no \textit{ad-hoc} assumptions.
Not even borrowing constraints.


\vspace{-1em}

\hypertarget{cfLiterature}{}
\hypertarget{DiffFromLit}{} 
\paragraph{Relationship to Literature}


Although the elements of buffer stock saving behavior were informally articulated by \cite{friedmanATheory}, the term was introduced to the literature by \cite{deatonLiqConstr} to describe the behavior of liquidity-constrained impatient consumers with transitory income shocks.\footnote{\cite{deatonLiqConstr} also showed that impatient consumers facing only permanent shocks would end up remaining on the borrowing constraint forever, an insight that informs the work of \cite{kvwWealthyH2m}.} \cite{carroll:brookings} showed (numerically) that buffer stock saving could arise even in the absence of borrowing constraints, and defined the individual buffer stock `target' as the point where a measure of normalized resources is expected to stay the same.


Traditional Bellman approaches to showing existence rely on assumptions that guarantee the boundedness of utility and marginal utility~\citep{slpMethods}.\footnote{Our CRRA utility function does not satisfy Bewley's assumption that $\uFunc(0)$ is well-defined, or that $\uP(0)$ is bounded above. %
  Our approach differs from that of Schectman and Escudero~\citeyearpar{seIncFluct} because they impose an artificial borrowing constraint and positive minimum income.
  It differs from Deaton~\citeyearpar{deatonLiqConstr} because he imposes liquidity constraints; we accommodate separate transitory and permanent shocks; and our transitory shocks occasionally cause income to reach zero.
  Papers by Scheinkman and Weiss~\citeyearpar{scheinkman&weiss:borrowing}, Clarida~\citep{claridaErgodic}, and others~\cite{cwcUnderUncert} all differ from ours for reasons resembling those articulated above.
%  For a related continuous-$t$ model, see~\cite{tocheUrisk}.
  \cite{asHomogeneous} relaxed boundedness of the utility function, but they address only the deterministic case; \cite{mvExistence}'s correction to~\cite{rrExistence} only addresses the deterministic case.
  \cite{mnUnique} assume a framework with compact action sets and real-valued utility which cannot handle relative risk aversion greater than 1. %
  
  Two approaches do allow relative risk aversion greater than 1:  A literature employing time iteration operators defined by Euler equations~\citep{deatonLiqConstr, lsIncFluct, mstIncFluct}, and one that employs transformations of the Bellman equation \citep{rinconZapatero2024}, but in all of these cases an artificial borrowing constraint is present (or its moral equivalent, as in Bewley). %
}.
The results by \cite{mstIncFluct,maUnboundedDP} are the most general we are aware of that tackle income fluctuation problems, and can be specialized to show existence in a model one step away from our normalized model with a stochastic rate of return and stochastic effective discount factor. The discrepant step is that they impose an artificial constraint and positive minimum value of income; this bounds utility from below (it can never be lower than the marginal utility of consumption)d thus cannot be applied here.

Our approach to constructing the weighted-norm space of value functions uses results on unbounded dynamic programming by \citep{jboydWeighted}.\footnote{\cite{asHomogeneous} showed how the approach could be used to address the homogeneous case (of which CRRA is an example) in a deterministic framework; later,~\cite{duranDiscounting} showed how to extend the~\cite{jboydWeighted} approach to the stochastic case.
  See also the exposition by \cite{stachurski2022}, Ch 12.} Our approach differs from previous approaches in its use of limiting marginal propensities to consume to construct per-period bounds on the Bellman operator.
Moreover, our patience restrictions are grounded in intuitive economic ideas (rather than abstract mathematical assumptions) that arise naturally in the presence of permanent income uncertainty and growth.
To the best of our knowledge, these economic mechanisms have not been explored elsewhere.

Our discussion of aggregate results builds on \cite{szeidlInvariant} and \cite{harmenbergInvariant} who give results on the existence and convergence of stationary wealth distributions that apply to the model presented here.
While their conditions for stationarity resemble \hyperlink{GIC}{growth impatience} and \hyperlink{GICMod}{strong growth impatience,} our objective is to establish existence of stable buffer stock targets, which is relatively easily tested empirically, rather than to establish stationarity of distributions, which is much harder to imagine testing with empirical data.


% documentclass is {article} or something else when main is compiling 
\ifSubfilesClassLoaded{ 
    % Content you want to compile only when standalone.
    \bibliography{\econtexRoot/\texname}
  }{}

\end{document}\endinput


% Local Variables:
% TeX-master-file: t
% eval: (setq TeX-command-list  (assq-delete-all (car (assoc "BibTeX" TeX-command-list)) TeX-command-list))
% eval: (setq TeX-command-list  (assq-delete-all (car (assoc "Biber"  TeX-command-list)) TeX-command-list))
% eval: (add-to-list 'TeX-command-list '("BibTeX" "bibtex %s" TeX-run-BibTeX nil t                                                                              :help "Run BibTeX") t)
% eval: (add-to-list 'TeX-command-list '("BibTeX" "bibtex %s" TeX-run-BibTeX nil (plain-tex-mode latex-mode doctex-mode ams-tex-mode texinfo-mode context-mode) :help "Run BibTeX") t)
% TeX-PDF-mode: t
% TeX-file-line-error: t
% TeX-debug-warnings: t
% eval: (advice-add 'TeX-command-expand :filter-return (lambda (command) (replace-regexp-in-string " -interaction=nonstopmode" "" command)))
% End:

% Local Variables:
% eval: (add-to-list 'TeX-expand-list '("mode" ""))
% End:

(defun my-tex-command-expand-options ()
    (let ((opts (TeX-command-expand-options)))
    (replace-regexp-in-string " -interaction=nonstopmode" "" opts)))

(setq TeX-expand-list-builtin
      (mapcar (lambda (item)
                (if (equal (car item) "mode")
                    '("mode" my-tex-command-expand-options)
                  item))
              TeX-expand-list-builtin))
